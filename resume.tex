\documentclass[10pt]{article}
\usepackage[a4paper, margin=1.25cm]{geometry}

\usepackage{array, xcolor}
\definecolor{lightgray}{gray}{0.8}
\newcolumntype{L}{>{\raggedleft}p{0.14\textwidth}}
\newcolumntype{R}{p{0.8\textwidth}}
\newcommand\VRule{\color{lightgray}\vrule width 0.10pt}
\pagenumbering{gobble}

\title{\Huge Abhishek Bagchi}
\author{abhishekbagchi@icloud.com}

\begin{document}

\maketitle

\vspace{0.5em}

\section*{Work Experience}
\begin{tabular}{L!{\VRule}R}
    2020-Present&\bf{Senior Engineer System Modelling}\\
    &\bf{ARM Ltd}\\
    &Performance modelling and initial investigation of future system architectures (C++)\\
    &Creating and maintaining a test infrastructure for systems modelling and analysis (Python)\\
    &Implementing and maintaining the modelling statistics infrastructure (C++)\\[10pt]

    2017-2020&\bf{Software Engineer}\\
    &\bf{ARM Ltd}\\
    &Modelling of AMBA AHB components (C++)\\
    &System level performance modelling and investigation (C++, Python)\\
    &Performance modelling and initial investigation of NoC architectures (C++)\\
    &Signal level to AMBA TML bridges for AMBA protocols (C++)\\[10pt]

    2015-2017&\bf{Graduate Software Engineer}\\
    &\bf{ARM Ltd}\\
    &Performance modelling and initial investigation of NoC architectures (C++, Python)\\
	&Performance modelling of master ports for L3 caches (C++)\\
    &Power management investigations using Gem5 (C++)\\[10pt]

    2014&\bf{Student Trainee}\\
    &\bf{Samsung R\&D Institute India}\\
    &Create a tool to comprehensively capture Android memory allocations and deallocations (Java, C++, Qt)\\
    &Bangalore, India\\[10pt]

    2012&\bf{Intern}\\
    &\bf{Cologne University Of Applied Sciences}\\
    &SkEtch3D (Java, OpenCV, ImageJ)\\
    &Cologne, Germany\\[10pt]
\end{tabular}

\section*{Education}
\begin{tabular}{L!{\VRule}R}
    2014-2015&\bf{Master of Sciences In Advanced Computing - Machine Learning, Data Mining and High Performance Computing}\\
    &University of Bristol\\[10pt]
    2010-2014&\bf{Bachelor of Engineering - Computer Science and Engineering}\\
    &Manipal Institute of Technology\\[10pt]
\end{tabular}

\section*{Master's Thesis}
\begin{tabular}{L!{\VRule}R}
    Title&\bf{Aligning video segments for appearance based navigation}\\
	Summary&This thesis aimed to tackle the problem of appearance based navigation. The intention was to investigate the suitability/practicality of navigation relying only on visual cues, and develop a prototype for a navigation system. Prototyped using Qt and OpenCV.\\[10pt]
\end{tabular}
\newpage

\section*{Projects}
\begin{itemize}
	\item \bf{Create a tool to comprehensively capture Android memory allocations and deallocations}
	\begin{itemize}
        \item[] \normalfont{Samsung R\&D Institute India - Bangalore}
        \item[] \normalfont{The project involved modifying Android(4.2) to allow logging memory allocations and deallocations, and creating a desktop application (using Qt and C++) that would capture those logs, and also allow the user to visualize the data.}
	\end{itemize}

	\item \bf{SkEtch3D}
	\begin{itemize}
        \item[] \normalfont{Cologne University Of Applied Sciences}
        \item[] \normalfont{The given task was to develop a module which would be given drawn natural sketches as input and convert them to spline curves. The project was done using Java, OpenCV and ImageJ.}
	\end{itemize}

	\item \bf{Simulate a superscalar processor}
	\begin{itemize}
        \item[] \normalfont{University of Bristol}
        \item[] \normalfont{Course project as a part of the course ‘Advanced Computer Architecture’ at the University of Bristol. The work involved designing and implementing a simulator for a superscalar processor. Implemented in Ada.}
	\end{itemize}

	\item \bf{KVDB}
	\begin{itemize}
        \item[] \normalfont{Personal project}
        \item[] \normalfont{Key value data store written in Golang. (github.com/AbhishekBagchi/kvdb)}
	\end{itemize}

	\item \bf{Pastebin-go}
	\begin{itemize}
        \item[] \normalfont{Personal project}
        \item[] \normalfont{Pastebin server written in Golang. (https://github.com/AbhishekBagchi/pastebin-go)}
	\end{itemize}
\end{itemize}

\section*{Volunteer Activities}
\begin{itemize}
	\item\bf{Mentor in the program EES Applied}
	\begin{itemize}
		\item[] Organisation: \normalfont{EDT (The Engineering Development Trust)}
		\item[] Mentor to a group of 16-18 year old students designing and prototyping projects for the blind set by the Royal National Institute of Blind People (RNIB).
	\end{itemize}
\end{itemize}

\section*{Computer Skills}
\begin{itemize}
    \item \bf{Programming Languages}: \normalfont{C++ (proficient), Python 2.7 (Intermediate), Go (Basics), JAVA (Basics), C (Basics)}
    \item \bf{Frameworks/Platforms}: \normalfont{GUI development using Qt framework for C++ , Application and Web Development using C\# on the .NET framework (prior experience)}
\end{itemize}

\section*{Organizational/Management Activities}
\begin{itemize}
	\item[] Board member of the students’ body, Institution of Engineers-Computer Science and Engineering, in capacity of Technical Head for the year 2012-13, responsible for providing technical direction while managing a team of over 90 students.
\end{itemize}

\end{document}
