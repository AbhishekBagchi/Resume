\documentclass[11pt]{article}
\usepackage[a4paper, margin=1.5cm]{geometry}

\usepackage{array, xcolor}
\definecolor{lightgray}{gray}{0.8}
\newcolumntype{L}{>{\raggedleft}p{0.14\textwidth}}
\newcolumntype{R}{p{0.8\textwidth}}
\newcommand\VRule{\color{lightgray}\vrule width 0.10pt}
\pagenumbering{gobble}

\title{\Huge Abhishek Bagchi}
\author{abhishekbagchi@icloud.com}

\begin{document}

\maketitle

\begin{minipage}[ht]{0.48\textwidth}
64, The Causeway\\
Soham, CB7 5BD\\
+447425761166
\end{minipage}

\vspace{2em}

\section*{Objective}
To work with pioneering organizations and be a part of a team which contributes to
the growth of the organization while simultaneously expanding my knowledge and skills.

\section*{Work Experience}
\begin{tabular}{L!{\VRule}R}
    2015-Present&\bf{Graduate Engineer}\\
    &\bf{ARM Ltd}\\
    &Initial modelling and investigation of NoC\\
    &Performance modelling using C++\\
	&Performance modelling of master ports for L3 caches using C++\\
    &Power management investigations using Gem5\\[10pt]

    2014&\bf{Student Trainee}\\
    &\bf{Samsung R\&D Institute India}\\
    &Bangalore, India\\[10pt]

    2012&\bf{Intern}\\
    &\bf{Cologne University Of Applied Sciences}\\
    &Cologne, Germany\\[10pt]
\end{tabular}

\section*{Education}
\begin{tabular}{L!{\VRule}R}
    2014-2015&\bf{Master of Sciences In Advanced Computing - Machine Learning, Data Mining and High Performance Computing}\\
    &University of Bristol\\[10pt]
    2010-2014&\bf{Bachelor of Engineering - Computer Science and Engineering}\\
    &Manipal Institute of Technology\\[10pt]
\end{tabular}

\section*{Master's Thesis}
\begin{tabular}{L!{\VRule}R}
    Title&\bf{Aligning video segments for appearance based navigation}\\
	Summary&This thesis aimed to tackle the problem of appearance based navigation. The intention was to investigate the suitability/practicality of navigation relying only on visual cues, and develop a prototype for a navigation system.\\[10pt]
\end{tabular}

\section*{Projects}
\begin{itemize}
	\item \bf{Create a tool to comprehensively capture Android memory allocations and deallocations}
	\begin{itemize}
        \item[] \normalfont{Samsung R\&D Institute India - Bangalore}
        \item[] \normalfont{The project involved modifying Android(4.2) to allow logging memory allocations and deallocations, and creating a desktop application(using Qt and C++) that would capture those logs, and also allow the user to visualize the data.}
	\end{itemize}

	\item \bf{SkEtch3D}
	\begin{itemize}
        \item[] \normalfont{Cologne University Of Applied Sciences}
        \item[] \normalfont{The given task was to develop a module which would be given drawn natural sketches as input and convert them to spline curves. The project was done using Java, OpenCV and ImageJ.}
	\end{itemize}

	\item \bf{Simulate a superscalar processor}
	\begin{itemize}
        \item[] \normalfont{University of Bristol}
        \item[] \normalfont{Course project as a part of the course ‘Advanced Computer Architecture’ at the University of Bristol. The work involved designing and implimenting a simulator for a superscalar processor. Done using Ada.}
	\end{itemize}

	\item \bf{Corporate Employee Welfare System}
	\begin{itemize}
        \item[] \normalfont{Done as a part of IBM’s The Great Mind Challenge, the task was to develop an internal social network for an organization allowing them to be share images, plan events, find solutions to technical problems etc. It was a web-based application, developed on Java Server Pages using Eclipse IDE and DB2 database.}
	\end{itemize}

	\item \bf{Elective subject Allotment system}
	\begin{itemize}
        \item[] \normalfont{Developed a website to automate the elective allotment system based on preference and grades. The website was developed using HTML and CSS for the client side and C\# and the ASP.NET framework for the server side}
	\end{itemize}

	\item \bf{Judiciary Information System}
	\begin{itemize}
        \item[] \normalfont{Developed a software to computerize certain aspects of the Judiciary system. The system was developed using C++ and the Qt framework. The database used was MySQL.}
	\end{itemize}
\end{itemize}

\section*{Volunteer Activities}
\begin{itemize}
	\item\bf{Mentor in the program EES Applied}
	\begin{itemize}
		\item[] Organisation: \normalfont{EDT (The Engineering Development Trust)}
		\item[] Act as a mentor to a group of 16-18 year old children who are working on projects set by the Royal National Institute of Blind People (RNIB).
	\end{itemize}
\end{itemize}

\section*{Computer Skills}
\begin{itemize}
    \item \bf{Programming Languages}: \normalfont{C++ (proficient), JAVA (Intermediate), Python 2.7 (Intermediate), C (Intermediate), Common LISP (Basics), Ada (Basics), VHDL (Prior Experience)}
	\item \bf{Frameworks/Platforms}: \normalfont{GUI development using Qt framework for C++ , Application and Web Development using C\# on the .NET framework}
\end{itemize}

\section*{Co-curricular Activities}
\begin{itemize}
	\item[] Conducted a workshop on ‘Core Java and Event-driven Programming Using Java’, ‘Application Development using Android’.
	\item[] Assisted in workshops on ‘Basics of Artificial Intelligence and implementation using C’, ‘Problem Solving Using Computers’, ‘Fundamental Data Structures using C’
	\item[] Headed the category ’System Admin’, which involved developing a web based system to allow registrations and managing participant data for Revels, the official cultural fest of Manipal Institute of Technology.
\end{itemize}

\section*{Organizational/Management Activities}
\begin{itemize}
	\item[] Board member of the students’ body, Institution of Engineers-Computer Science and Engineering, in capacity of Technical Head for the year 2012-13, responsible for a team of over 90 students.
\end{itemize}

\end{document}
